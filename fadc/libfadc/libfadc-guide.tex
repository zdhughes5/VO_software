\documentclass{article}
\usepackage{fancyhdr}
\pagestyle{fancy}

%=====================================================
% macros

\newcommand{\libfadc}{\emph{\texttt{libfadc}} }
\newcommand{\F}[1]{\texttt{#1}}

%=====================================================

\begin{document}

  

  \title{Usage Guide for \libfadc}
  \author{Karl Kosack}
  \maketitle
  \tableofcontents

\section{Introduction}

The \libfadc library provides a straightforward C interface
\footnote{The C language was chosen initially for portability between
QNX4 and Linux. Of course, all functions will work fine in C++ as
well. If it turns out to be absolutely necessary, I will convert the
library to C++.} to an FADC crate (several FADC boards plus a
Clock/Trigger board). Currently, \libfadc takes care of the VME
mapping, all the low-level board access, and retains the state of each
board in memory (board settings, etc).  For a list of the \libfadc API
functions, see the Doxygen generated documentation (\libfadc reference
guide).

\section{A simple example}

Suppose we want to initialize a crate, do a buffer ram test to make
sure the boards are working, set some parameters, and then grab a few
events of data.  The following code would accomplish this:

First, we start the programs and call \F{fadc\_init()} to set up the
VME interface.  \F{fadc\_init()} must be called before all other
\libfadc functions (except \F{fadc\_set\_initial\_values()} and \F{fadc\_ignore\_board()}).

\begin{verbatim}
#include <stdio.h>
#include <fadc.h>

int main( void ) {
  
  int i,n;
  unsigned long *cblt_buffer=NULL;

  printf("Initializing %d FADC boards...\n", fadc_num_boards() );

  if ( fadc_init() ) {
    printf( "Failed to initialize.\n");
    exit(1);
  }

\end{verbatim}

Next, we put the FADC boards into WORD mode (disable all acquisition,
but still accept commands), and test the buffer ram...

\begin{verbatim}

  fadc_set_mode( FADC_ALL, WORD_MODE );

  for (i=0; i<fadc_num_boards(); i++) {
    if (fadc_buffer_ram_test(i)) {
      printf("Board %d failed the buffer ram test!\n",i);
      fadc_exit();
      exit(1);
    }
  }
\end{verbatim}

Finally, we set the lookback times (say to 120 ns for all boards and
all channels), and some other parameters. Note that this could be done
all at once using the \F{fadc\_load\_settings()} function, which
reads all settings from a text file, but we may want to do it some
other way. We can of course specify a specific board and channel
instead of FADC\_ALL if needed.

\begin{verbatim}

  fadc_set_area_offset( FADC_ALL, FADC_ALL, 120 );
  fadc_set_data_width( FADC_ALL, 16 );

  ...

\end{verbatim}

Now, lets grab some events (10 in this example).  For that, we need to
first put the boards into data acquisition mode (FADC\_MODE) and then
poll the Clock/Trigger board for status (using
\F{fadc\_got\_event()}).  When an event is received, each board with
data should be read out.  This can be done in two ways: using a
word-by-word transfer (using \F{fadc\_memcpy()}), or using a
\emph{Chained Block Transfer} (CBLT).  The CBLT method allows much
faster data rates, and all boards in the crate are read out in one
pass, so lets look at that one.

\begin{verbatim}

  n=0;
  int nwords=0;

  /* Allocate memory for the CBLT buffer and initialize CBLT to use
  board 0 as the 'first' board and  9 as the 'last' board */
  
  cblt_buffer = fadc_alloc_cblt_buffer( cblt_buffer, 0, 9 );

  while(n < 10) {
    
    if ( fadc_got_event() ) {

      /* we have an event, so read out the data */
    
      nwords = fadc_cblt();

      /* Now the cblt_buffer is filled with nwords words of data */
      
      /* done reading, so issue an "event clear" */
      /* to the Clock/Trigger board so the FADC's keep taking data */
      
      fadc_evt_clr();

      n++;

      /* do something with the buffer */
      ...
      
    }
  }

  /* All done! */
  fadc_exit();

}

\end{verbatim}

\section{Other Comments} 

\subsection{Disabling Boards}

When \F{fadc\_init()} is called, it automatically initializes an
entire crate of FADC boards (the total number of boards is defined in
the library).  However, you may only want to initialize a few of the
boards.  To do this, you need to tell \libfadc to ignore the inactive
boards.

\begin{verbatim}

  /* only enable board 2 in the crate */
  for (i=0; i<fadc_num_boards(); i++){
    if (i != 2) fadc_ignore_board(i);
  }
\end{verbatim}

\subsection{Disabling channels}

You can disable a particular channel on a board using
\F{fadc\_disable\_channel(board,chan)} and check whether it is
disabled with \F{fadc\_is\_channel\_disabled(board,chan)}

\subsection{Loading/Saving all settings}

The functions \F{fadc\_load\_settings(board,filename)} and
\F{fadc\_save\_settings(board,filename} allow you to easily save or
load all of the settings for a particular board in one quick step,
rather than setting each parameter individually.  The settings are
saved in text format the specified file.


\section{Low-level access}

In addition to the high-level API, \libfadc also provides low-level
access to the FADC and Clock/Trigger boards.  This includes a set of
functions for writing to the various registers on the boards and for
direct access to the memory-mapped-io hardware address space for each
board.  To use these functions, you must include \F{fadc\_lowlevel.h}
in your program.  See the \libfadc Reference Guide for a function
reference.

For example, to get a pointer to the beginning of the address space
for FADC board 3 and read a few words of the buffer ram, do the
following:

\begin{verbatim}

  unsigned long *lptr;
  unsigned long dataword;
  int i;

  lptr = fadc_get_fadc_lptr( 3 );

  for (i=0; i<5; i++) {
    
    dataword = lptr[i];
    cout << "Buffer " << i << " = " << dataword << endl; 

  }

\end{verbatim}

Or to write to a register (for instance to change the programmable
board number), you would do the following:

\begin{verbatim}

  /* Set programmable board number on board 0 to be 2 */
  /* The offset WR_BD_NO is defined in fadc_lowlevel.h */

  unsigned long *fadc_lptr, *boardnum_register; 

  fadc_lptr = fadc_get_fadc_lptr( 0 );
  
  boardnum_register = fadc_lptr + WR_BD_NO;
  *boardnum_register = 2;

\end{verbatim}




\end{document}
